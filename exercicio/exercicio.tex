% Exercício - LaTeX para Teses e Dissertações do INPE
% II Semana de Imersão da ABPG
% 06 de Julho de 2021
% carlos.bastarz@inpe.br (@cfbastarz)

% A diretiva "documentclass" informa para o LaTeX qual é a classe do documento a ser produzido;
% Observe que as opções da diretiva, são informadas entre os colchetes "[]".
% Outras classes de documento são: "article", "book", "report", "letter"
\documentclass[10pt]{article}
%\documentclass[10pt]{book}
%\documentclass[10pt]{report}
%\documentclass[10pt]{letter}

% Em um documento LaTeX, comentários são precedidos por "%" e podem ser inseridos em qualquer parte do documento.

% Este é o preâmbulo (tudo que está entre as diretivas "documentclass" e "\begin{document}").

% O pacote "inputenc" fornece opções de codificação relacionadas com a representação dos caracteres;
% A opção "utf8" permite a inserção de caracteres acentuados.
% O pacote "inputenc" não precisa ser carregado quando o compilador da linguagem for o XeLaTeX ou o LuaLaTeX
\usepackage[utf8]{inputenc}

% O pacote "babel" fornece opções de localização do documento;
% A opção "brazilian" localiza o documento para o idioma Português do Brasil (permite traduzir rótulos como Figure, Table, Contents etc. para Figura, Tabela e Sumário).
\usepackage[brazilian]{babel}

% O pacote "lipsum" permite gerar texto prolixo.
\usepackage{lipsum}

% O pacote "booktabs" fornece opções para a diagramação de tabelas.
\usepackage{booktabs}

% O pacote "amsmath" fornece opções para a inserção de equações.
\usepackage{amsmath}

% O pacote "graphicx" permite utilizar imagens de exemplo.
\usepackage{graphicx}

% O pacote "subfigure" permite inserir um figuras lado a lado dentro do ambiente "figure".
\usepackage{subfigure}

% O pacote "float" fornece opções para o posicionamento de corpos flutuantes (e.g., Figuras e Tabelas).
\usepackage{float}

% O pacote "abntex2cite" fornece opções de referenciação de acordo com as normas da ABNT;
% A opção "alf" insere as referências no padrão alfanumérico.
\usepackage[alf]{abntex2cite}

% O pacote "bibentry" permite a inserção de referências no formato BibTeX.
\usepackage{bibentry}

% Macros são conjuntos de instruções encapsuladas do LaTeX. A macro "baselinestretch" define o espaçamento entre as linhas em um documento e o seu valor padrão é 1.0.
% O comando "renewcommand" é utilizado para redefinir macros do LaTeX;
% O comando a seguir redefine o valor da macro "baselinestretch" para 1.3 (equivalente ao espaçamento médio);
%\renewcommand{\baselinestretch}{1.0} % é o espaçamento simples (padrão)
\renewcommand{\baselinestretch}{1.3} % é o espaçamento médio
%\renewcommand{\baselinestretch}{1.6} % é o espaçamento duplo

% Título, autor e data do documento
\title{\LaTeX para Teses e Dissertações do INPE \\ II Semana de Imersão da ABPG}
\author{Carlos Frederico Bastarz}
\date{\today}

% O conteúdo do documento, é inserido entre as diretivas "\begin{document}" e "\end{document}"
\begin{document}

\setlength{\parindent}{3em} % recuo dos parágrafos
\setlength{\parskip}{1em} % espaçamento entre os parágrafos

\maketitle

%\tableofcontents

\section*{Resumo}

\lipsum[1]

\section{Introdução}

\lipsum[2-3]

\section{Material e Métodos}

\lipsum[4]

\begin{itemize}
    \item Item;
    \item Item;
    \begin{itemize}
        \item Item;
        \item Item;
        \begin{itemize}
            \item Item;
            \item Item;
        \end{itemize}
    \end{itemize}
    \item Item;
    \item Item;
\end{itemize}

\lipsum[5]

\begin{table}[h]
    \label{tab:my_table}
    \centering
    \caption{Uma tabela de exemplo.}
    \begin{tabular}{c|c|c}
        \toprule
        \textbf{L1C1} & \textbf{L1C2} & \textbf{L1C3} \\
        \midrule
        L2C1          & L2C2          & L2C3 \\
        L3C1          & L3C2          & L3C3 \\
        L4C1          & L4C2          & L4C3 \\
        \bottomrule
    \end{tabular}
\end{table}

\lipsum[6]

\begin{multline} 
    A(x,y)\frac{\partial^2{\Psi}}{\partial{x^2}} +
    B(x,y)\frac{\partial^2{\Psi}}{\partial{x}\partial{y}} +
    C(x,y)\frac{\partial^2{\Psi}}{\partial{y^2}} +
    D(x,y)\frac{\partial{\Psi}}{\partial{x}} + \\ +
    E(x,y)\frac{\partial{\Psi}}{\partial{y}} + F(x,y)\Psi = G(x,y)
\end{multline}

\lipsum[7]

Segundo \citeonline{ciclanoetal/1975,fulano/1964}, a ciência é comunicada e registrada através de artigos, relatórios, apresentações etc. A ciência é comunicada e registrada através de artigos, relatórios, apresentações etc. \cite{ciclanoetal/1975,fulano/1964}.

\section{Resultados e Discussão}

\lipsum[8]

\begin{figure}[H]
    \centering
    \includegraphics[scale=0.6]{example-image-a}
    \caption{Uma figura de exemplo.}
    \label{fig:my_label1}
\end{figure}

\lipsum[9-10]

\begin{figure}[H]
    \centering
    \subfigure[Figura A]{\includegraphics[scale=0.49]{example-image-a}}
    \subfigure[Figura B]{\includegraphics[scale=0.49]{example-image-b}}
    \subfigure[Figura C]{\includegraphics[scale=0.49]{example-image-c}}
    \caption{Três figuras de exemplo.}
    \label{fig:my_label2}
\end{figure}

\lipsum[11-11]

\bibliography{referencias}

\end{document}